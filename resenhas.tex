%%
%% Copyright (c) 2001, 2009, 2010 The American Physical Society.
%%
%% See the REVTeX 4 README file for restrictions and more information.
%%
%To achieve the polarity reversal, several techniques have already been employed:
%magnetic field bursts6, oscillating perpendicular magnetic fields7 and %in-plane rotating magnetic field8, among others. We decided to employ a %rotating magnetic field due to the low intensity of the applied field and %the frequency selectivity available in the process, i.e., the vortex core %will only reverse for a well-defined range of field frequencies12. It was %also noted that the gyrotropic frequency decreased for increasing values of %K.


% This is a template for producing manuscripts for use with REVTEX 4.0
% Copy this file to another name and then work on that file.
% That way, you always have this original template file to use.
%
% Group addresses by affiliation; use superscriptaddress for long
% author lists, or if there are many overlapping affiliations.
% For Phys. Rev. appearance, change preprint to twocolumn.
% Choose pra, prb, prc, prd, pre, prl, prstab, prstper, or rmp for journal
% Add 'draft' option to mark overfull boxes with black boxes
% Add 'showpacs' option to make PACS codes appear
% Add 'showkeys' option to make keywords appear
%\documentclass[aps,prb,showpacs,reprint,groupedaddress]{revtex4-1}
%\documentclass[aps,showpacs,prl,preprint,superscriptaddress]{revtex4-1}
% Originalmente estava assim:
%\documentclass[nofootinbib,aps,reprint,superscriptaddress]{revtex4}
% Eu resolvi deixar assim:
\documentclass[hidelinks,a4paper,nofootinbib,aps,reprint,superscriptaddress]{revtex4}
\linespread{1.4}

%\documentclass[showpacs,aip,apl,twocolumn,groupedaddress]{revtex4-1}
%\documentclass[aps,showpacs,apl,twocolumn,reprint,groupedaddress]{revtex4-1}
%\documentclass[nature,twocolumn,reprint,groupedaddress]{revtex4-1}
% You should use BibTeX and apsrev.bst for references
% Choosing a journal automatically selects the correct APS
% BibTeX style file (bst file), so only uncomment the line
% below if necessary.
%\bibliographystyle{apsrev4-1}
\bibliographystyle{abntex2-num}

\usepackage{graphicx}
%\usepackage{mathscr}
\usepackage{amsmath,amsfonts}
\usepackage{xcolor}
\usepackage{lineno}
\definecolor{Red}{rgb}{0.9,0.0,0.1}
\definecolor{Blue}{rgb}{0.1,0.1,0.9}
%\usepackage{tabulary}
%\usepackage{subfigure}%
\hyphenation{ma-the-ma-tics e-qui-li-bri-um Bourdieu Ginzburg Adorno Lacey Bradbury Latour Mauss Rosenberg}

\usepackage[brazil]{babel}

\usepackage[utf8]{inputenc}
\usepackage[T1]{fontenc}

\usepackage{natbib}
\usepackage[autostyle]{csquotes}  

\usepackage{hyperref}
\hypersetup{
	pdftitle    = {Resenhas},
	pdfsubject  = {Metropolização},
	pdfauthor   = {Caio César Carvalho Ortega},
	pdfcreator  = {Caio César Carvalho Ortega},
	pdfproducer = {Caio César Carvalho Ortega},
	pdfkeywords = {metropolização, metrópole, governança metropolitana}
}


\makeatletter
\newcommand*{\citenst}[2][]{%
	\begingroup
	\let\NAT@mbox=\mbox
	\let\@cite\NAT@citenum
	\let\NAT@space\NAT@spacechar
	\let\NAT@super@kern\relax
	\renewcommand\NAT@open{[}%
	\renewcommand\NAT@close{]}%
	\cite[#1]{#2}%
	\endgroup
}
\makeatother

\listfiles

\begin{document}
	%\linenumbers
	% Use the \preprint command to place your local institutional report
	% number in the upper righthand corner of the title page in preprint mode.
	% Multiple \preprint commands are allowed.
	% Use the 'preprintnumbers' class option to override journal defaults
	% to display numbers if necessary
	%\preprint{}
	
	%Title of paper
	\title{Resenhas para a disciplina de Política Metropolitana}
	
	% repeat the \author .. \affiliation etc. as neededcitacao
	% \email, \thanks, \homepage, \altaffiliation all apply to the current
	% author. Explanatory text should go in the []'s, actual e-mail
	% address or url should go in the {}'s for \email and \homepage.
	% Please use the appropriate macro foreach each type of information
	
	% \affiliation command applies to all authors since the last
	% \affiliation command. The \affiliation command should follow the
	% other information
	% \affiliation can be followed by \email, \homepage, \thanks as well.
	%\author{}
	
	\affiliation{Universidade Federal do ABC, Centro de Engenharia, Modelagem e Ciências Sociais Aplicadas, São Bernardo do Campo-SP, Brasil}
	
	\author{Caio César Carvalho Ortega, RA 21038515}
	
	%\date{\today}
	
	\maketitle

	% ----------------------------------------------------------
	
	\section{Prólogo}
	
	O propósito do presente trabalho é realizar três breves resenhas para a disciplina de Política Metropolitana (ESHT020), constituídas de um parágrafo cada.
	
	\section{Resenhas}
	
	\subsection{Primeira Resenha}
	
	O artigo aborda a problemática da criação de regiões metropolitanas no Brasil, a luz sobretudo do Estatuto da Metrópole, sem desprezar o marco regulatório anterior, citado para efeito de comparação e contextualização histórica, contribuindo para elucidar o papel da União (conforme os artigos 21, XX e 24, I da Constituição Federal), reforçado não só pela tramitação do Estatuto da Metrópole no ano de 2004, que visa ``tratar da criação das regiões metropolitanas, aglomerações
	urbanas e microrregiões'', mas também pela ``edição da Lei no 11.445 em 2007, que estabeleceu as diretrizes nacionais para o saneamento 	básico; a Lei da Política Nacional de Resíduos Sólidos (Lei Federal nº 9.605/2010); as diretrizes da 	Política Nacional de Mobilidade Urbana (Lei Federal nº 12.587/2012)'' \cite[p.3]{mencio2017}. Antes do Estatuto da Metrópole a criação de regiões metropolitanas estava totalmente dependente de leis estaduais, como aquela que reorganizou a Região Metropolitana de São Paulo - RMSP após a redemocratização (Lei Complementar nº 1.139 de 16 de junho de 2011), visto que originalmente a RMSP havia sido criada pelo Governo Federal (Lei Federal Complementar nº 14 de 1973). Considero de suma importância a avaliação de que o Estatuto da Metrópole passa a considerar não mais a conurbação, mas ``a existência de unidade territorial constituída por no mínimo dois municípios limítrofes, caracterizada por complementariedade funcional e integração das dinâmicas geográficas, ambientais, políticas e socioeconômicas, que apresenta destacada influência nacional ou regional, conforme critérios adotados pelo IBGE'' \cite[p.5]{mencio2017}, chamada de \textbf{aglomeração urbana}. O artigo cita doutrinas jurídicas relacionadas ao tema, sendo valiosa a menção daquelas que fortalecem o IBGE e as diretrizes do Estatuto da Metrópole à época de publicação do \textit{paper}, sendo considerado que ``a União participa do planejamento urbano nacional, conferindo as diretrizes gerais para os planejamentos regionais e locais'' \cite[p.18]{mencio2017}, não existindo assim uma violação da autonomia dos estados e/ou municípios (não há violação do art. 25, §3º da Constituição Federal), considerando que o diploma federal contribuirá para reduzir as disparidades urbanas, fruto da metropolização, para tanto estabelecendo segurança jurídica e planejamento urbano uniforme em nível nacional, a partir do estabelecimento de parâmetros gerais (ou seja, \textbf{a União não está obrigando}, apenas condiciona a criação de figuras regionais aos parâmetros, o legislador estadual pode simplesmente optar por não instituir as figuras). Finalmente, o Estatuto da Metrópole não é necessariamente uma novidade, pois ``algumas leis estaduais, entre elas a do Estado de Minas Gerais, já apresentavam este conteúdo como forma de subsidiar o exercício da função legislativa no ato da criação das figuras regionais'' \cite[p.20]{mencio2017}.
	
	\subsection{Segunda Resenha}
	
	O autor aponta ainda no início as limitações do marco regulatório brasileiro para regiões metropolitanas, que não está a altura da complexidade e dos desafios do tema, assim também indicando que na altura estávamos a testemunhar internacionalmente um ``processo efervescente e diversificado de experimentação na gestão metropolitana'' \cite[p.416]{klink2009}. O autor elabora uma reflexão sobre como as experiências internacionais podem contribuir para a gestão metropolitana no Brasil, citando, por exemplo, o caso da Comunidade Autônoma Madrilense; o autor considera que as figuras regionais autônomas da região metropolitana de Madrid constituem um caso excepcional, dado o elevado grau de simetria entre a região funcional e a administrativa/institucional, no entanto, aponta também a inflexão devido aos conflitos no âmbito social e institucional pela participação de pactuação da escala metropolitana, com duplicação de esforços por parte do governo central de Madrid (o que fragiliza o processo de planejamento do território) e aumento das reivindicações da sociedade civil, já o modelo norte-americano é considerado frágil, sendo citado o caso de Nova Iorque, cuja região metropolitana ``espalha-se por \textbf{3 estados, 31 condados, 800 municípios e mais de 1.000 distritos específicos} voltados para a provisão de serviços setoriais de interesse comum'' \cite[p.418]{klink2009} (grifo meu), destacando uma problemática que envolve segregação, racismo e um conjunto abrangente de escalas e atores. Ao discutir o caso brasileiro, o autor cita o marco regulatório a partir dos anos 1970, os desafios e o esgotamento do modelo encarados nos anos 1980 com a crise fiscal que assolou o Estado Brasileiro e a proliferação de arranjos institucionais estadualizados na década de 1990. É preciso destacar a crítica feita pelo autor à Constituição Federal de 1988: ``os novos atores sociais associaram o tema metropolitano ao regime militar e não o pautaram com a devida atenção e consistência no processo constituinte de 1988'' \cite[p.419]{klink2009}, resultando numa forte delegação aos estados por parte da União quanto à gestão das figuras regionais de âmbito metropolitano, processo este também relacionado ao surgimento de arranjos horizontais de associativismo intermunicipal (que decorreu já anos 1980, sendo impulsionado por figuras como o ex-governador paulista Franco Montoro, entusiasta da ideia de consórcios). A União volta a olhar para as metrópoles em 2003, com os ministérios das Cidades e da Integração Nacional e a Subchefia de Assuntos Federativos da Casa Civil da Presidência da República. O autor conceitua que existem dois tipos de arranjos governamentais: (i) arranjo governamental de múltiplas escalas e	(ii) arranjo governamental intermunicipal. Finalmente, o autor defende uma maior participação da esfera federal, como articuladora e orientadora, defendendo ainda que o governo federal \textbf{não esvazie} o papel do estado e criticando oportunidades perdidas pela União, apontando que esta ``deixou de aproveitar um conjunto de instrumentos financeiros de fomento à pactuação metropolitana'' \cite[p.428]{klink2009}.
	
	\subsection{Terceira Resenha}
	
	Inicialmente o autor destaca o protagonismo pujante das metrópoles no capitalismo, ponderando que sua importância também representa grandes desafios, o que pode ser resumido pela colocação a seguir: ``(\dots) as metrópoles têm dificuldade em se tornarem verdadeiros territórios políticos, dimensão ao mesmo tempo necessária e constitutiva de sua governabilidade'' \cite[p.300]{lefevre2009}. Para o autor, considerando o contexto europeu e excetuando-se Madrid, a metrópole é ``uma aporia da descentralização'' \cite[p.301]{klink2009}, apontando uma espécie de limbo institucional em países como Itália (Lei 142, que cria as cidades metropolitanas nunca executada) e a falta de avanços legislativos na Alemanha, Holanda e Reino Unido, o que contrasta com a França, que desde 1999 possui uma lei acerca da ``intercomunabilidade que cria comunidades de aglomeração'' \cite[p.301]{klink2009}. O autor cita ainda referendos em Amsterdã, Roterdã e Berlim, feitos em meados dos anos 1990 e que resultaram na rejeição da criação de uma autoridade metropolitana. Ainda que os debates no âmbito acadêmico sejam muitos, a Europa tem buscado uma administração funcional, nada mais. O debate é permeado pela multiplicidade de pontos de vista, sem que necessariamente um seja vencedor. O autor cita o caso de Paris e Turim como positivos, por existir um processo amplo de articulação de múltiplos atores. O autor cita que a descentralização não contribui para dar vazão a uma ordem institucional que consagre o fato metropolitano politicamente. Finalmente, salienta que os regimes existentes, ligados à coletividades ou negócios, produzem resultados diferentes.
	
	% ----------------------------------------------------------
	
	\bibliography{fontes.bib}
	
\end{document}


